\chapter{Klaszterezés vagy osztályozás}

A tisztított adathalmazon \textbf{osztályozási} feladatot valósítottam meg: 
a cél a címek \textit{Movie} (0) és \textit{TV Show} (1) kategóriába sorolása volt.

\Section{Modell célja és választott módszer}
A feladat bináris, jól értelmezhető célváltozóval rendelkezik (\texttt{type\_norm}), 
ezért felügyelt tanulási megközelítést választottam. 
A modell egy \textbf{logisztikus regresszió}, amely lineáris döntési határt tanul, 
robosztus, jól magyarázható és gyorsan tanítható nagyobb mintákon is.

\Section{Jellemzők és előfeldolgozás}
A modell a következő jellemzőket használta:
\begin{itemize}
	\item \textbf{Numerikus:} \texttt{release\_year}, \texttt{duration\_minutes}, \texttt{seasons}, 
	\linebreak
	\texttt{cast\_count};
    \item \textbf{Kategóriális:} \texttt{rating\_clean}, \texttt{country\_main};
    \item \textbf{Bináris műfaj-indikátorok:} a 20 leggyakoribb műfaj (\texttt{genre\_*} oszlopok).
\end{itemize}

Az előfeldolgozás \textit{pipeline}-ban történt:
\begin{enumerate}
    \item A numerikus jellemzők hiányzó értékeit mediánnal pótoltam, majd standardizáltam a \texttt{StandardScaler} segítségével.
    \item hiányzó kategóriák pótlása leggyakoribb értékkel, \texttt{OneHotEncoder} alkalmazása;
    \item a bináris műfajoszlopok hiányzó értékeinek 0-ra imputálása.
\end{enumerate}
A tanuló- és teszthalmazt \textbf{stratifikált} módon választottam szét (75\%--25\%).

\Section{Teljesítménymutatók}
A modellt a teszthalmazon értékeltem. A legfontosabb mutatók:

\begin{table}[H]
\centering
\begin{tabular}{lcc}
\toprule
\textbf{Mutató} & \textbf{Érték} \\
\midrule
Pontosság (Accuracy) & $\approx 0.95$ \\
ROC--AUC & $\approx 0.97$ \\
\bottomrule
\end{tabular}
\caption{Az osztályozó teljesítménymutatói a teszthalmazon}
\end{table}

A részletes osztályonkénti precízió, visszahívás és F1-értékek a kimeneti jelentésben szerepelnek; 
összességében a modell mindkét osztályt magas pontossággal azonosítja.

\Section{Konfúziós mátrix}
A helyesen és tévesen besorolt esetek megoszlását a konfúziós mátrix szemlélteti.

\begin{figure}[H]
    \centering
    \includegraphics[width=0.75\textwidth]{images/plot_confusion_matrix.png}
    \caption{Konfúziós mátrix (0 = Movie, 1 = TV Show)}
\end{figure}

\Section{Modell alkalmazása új adatokon}
A betanított modellt „új” megfigyeléseken (a félretett teszthalmaz 5 véletlen rekordján) is kipróbáltam. 
Minden címhez a modell nemcsak a címkét adja meg (\textit{Movie}/\textit{TV Show}), 
hanem a valószínűséget is ($P(\text{TV Show})$). 
A predikciók jól értelmezhetők: a sorozatoknál jellemzően magas a \texttt{seasons} és egyes műfaj-indikátorok értéke, 
míg a filmeknél a \texttt{duration\_minutes} informatív.

\Section{Megjegyzések és továbbfejlesztés}
A logisztikus regresszió erős kiindulópont, ugyanakkor további javulás érhető el:
\begin{itemize}
    \item alternatív modellek kipróbálásával (pl. \textit{Random Forest}, \textit{XGBoost});
    \item keresztvalidációval és hiperparaméter-hangolással;
    \item szöveges mezők (pl. \texttt{description}) \textit{TF--IDF} reprezentációjának bevonásával.
\end{itemize}
Mindazonáltal a jelenlegi eredmények alapján a modell megbízhatóan különbözteti meg a filmeket és a sorozatokat.
