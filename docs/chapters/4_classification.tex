\chapter{Klaszterezés vagy osztályozás}

A tisztított adathalmazon \textbf{osztályozási} feladatot valósítottam meg:
a cél a címek \textit{Movie} (0) és \textit{TV Show} (1) kategóriába sorolása volt.
A probléma bináris jellegű, ezért felügyelt tanulási megközelítést alkalmaztam.

\Section{Modell célja és választott módszer}

A célváltozó (\code{type\_norm}) jól értelmezhető, kétértékű mező,
így természetes választás volt a \textbf{logisztikus regresszió}.
A modell előnye, hogy:

\begin{itemize}
    \item lineáris döntési határt tanul;
    \item a becsült együtthatók értelmezhetők (mely jellemzők növelik a \newline TV~Show valószínűségét);
    \item gyorsan és stabilan tanítható;
    \item robusztus nagy számú bináris jellemző esetén (például műfajok).
\end{itemize}

\Section{Numerikus jellemzők korrelációja}

Az alábbi ábra azt mutatja meg, hogy a fő numerikus jellemzők 
\newline
--\code{release\_year}, \code{duration\_minutes}, \code{seasons},
\code{cast\_count} -- \newline -- hogyan függnek össze egymással.

Az értékek:
\begin{itemize}
    \item \textbf{1.0} -- teljes korreláció (pl.\ egy változó önmagával),
    \item \textbf{0} -- nincs kapcsolat,
    \item \textbf{negatív} -- ellentétes irányú kapcsolat,
    \item \textbf{pozitív} -- együtt növekednek.
\end{itemize}

A mátrix alapján például:
\begin{itemize}
    \item a \code{duration\_minutes} és \code{cast\_count} között mérsékelt pozitív kapcsolat figyelhető meg;
    \item a \code{release\_year} és \code{duration\_minutes} között gyenge negatív korreláció van
          (az újabb tartalmak jellemzően rövidebbek);
    \item a \code{seasons} változó a többihez képest gyenge kapcsolatot mutat,
          ami érthető, hiszen csak sorozatoknál értelmezhető.
\end{itemize}

\begin{figure}[H]
    \centering
    \includegraphics[width=0.7\textwidth]{images/plot_numeric_correlations.png}
    \caption{A numerikus jellemzők közötti korreláció}
\end{figure}

\Section{Felhasznált jellemzők és előfeldolgozás}

A modell az alábbi jellemzőcsoportokra épült:

\begin{itemize}
    \item \textbf{Numerikus:} \code{release\_year}, \code{duration\_minutes}, \code{seasons}, \newline
     \code{cast\_count};
    \item \textbf{Kategóriális:} \code{rating\_clean}, \code{country\_main};
    \item \textbf{Bináris műfaj-indikátorok:} a 20 leggyakoribb műfaj (\code{genre\_*}).
\end{itemize}

Az előfeldolgozást \textit{pipeline} segítségével valósítottam meg:

\begin{enumerate}
    \item a numerikus jellemzők hiányzó értékeinek pótlása mediánnal, majd standardizálás;
    \item a kategóriák hiányzó értékeinek pótlása és \code{OneHotEncoder} alkalmazása;
    \item a bináris műfajmezők hiányzó értékeinek 0-ra pótlása.
\end{enumerate}

A tanuló- és teszthalmazt stratifikált módon választottam szét 75\%--25\% arányban.

\Section{Modellparaméterek}

A logisztikus regresszió konfigurációja:

\begin{itemize}
    \item \code{solver='lbfgs'};
    \item \code{max\_iter=400};
    \item \code{penalty='l2'} (alapértelmezett regulárizáció).
\end{itemize}

A betanított modell koefficiensei alapján:

\begin{itemize}
    \item a \textbf{seasons} és számos műfajindikátor növeli a TV~Show valószínűségét;
    \item a \textbf{duration\_minutes} erősen negatív hatású (a rövidebb tartalmak jellemzően sorozatok);
    \item a \textbf{release\_year} pozitív előjelű (újabb tartalom $\rightarrow$ nagyobb esély sorozatra).
\end{itemize}

\Section{Modell teljesítménye}

A teszthalmazra kapott legfontosabb mérőszámok:

\begin{table}[H]
\centering
\begin{tabular}{lc}
\toprule
\textbf{Mutató} & \textbf{Érték} \\
\midrule
Pontosság (Accuracy) & $0.95$ \\
ROC--AUC & $0.97$ \\
\bottomrule
\end{tabular}
\caption{Az osztályozó teljesítménymutatói a teszthalmazon}
\end{table}

A részletes osztályonkénti precízió, visszahívás és F1 értékek a Python-kimenetben szerepelnek.

\Section{Konfúziós mátrix}

A helyesen és hibásan besorolt rekordok megoszlását a következő ábra mutatja:

\begin{figure}[H]
    \centering
    \includegraphics[width=0.75\textwidth]{images/plot_confusion_matrix.png}
    \caption{Konfúziós mátrix (0 = Movie, 1 = TV~Show)}
\end{figure}

\Section{Téves besorolások elemzése}

A modell a teszthalmazban néhány határesetet hibásan sorolt be.
Az alábbi példák a legjellemzőbb eseteket mutatják:

\begin{itemize}
    \item \textbf{Bob Ross: Beauty Is Everywhere} -- rövid, epizodikus felépítése miatt nehezen különíthető el egyes dokumentumfilmekhez hasonló tartalmaktól.
    \item \textbf{Warrior Nun} -- csak egy évada van, és a játékidő hiányzik az adatokból, ezért hasonlónak tűnhet egy hosszabb filmhez.
    \item \textbf{Metallica: Some Kind of Monster} -- dokumentumfilm, de a platform sorozatként jelöli; a hiányzó időtartam megzavarhatta a modellt.
    \item \textbf{American Masters: Inventing David Geffen} -- film, de több részes dokumentumfilmekhez hasonló besorolási mintázatai lehetnek.
\end{itemize}

\Section{5-fold Cross Validation eredménye}

A modell stabilitásának vizsgálatához 5-szörös keresztvalidációt alkalmaztam.
Az eredmények:

\begin{itemize}
    \item Fold pontosságok: 0.991--0.999 tartományban;
    \item Átlagos pontosság: \textbf{0.9965}.
\end{itemize}

Ez azt mutatja, hogy az osztályozó rendkívül stabil,
és nem függ erősen a mintavételtől vagy a konkrét tanulóhalmaztól.

\Section{Modell alkalmazása új adatokon}

A modellt néhány véletlenszerűen kiválasztott teszthalmazbeli példán is kipróbáltam.
A modell minden címhez nemcsak az előrejelzett kategóriát adja meg (Movie / TV~Show),
hanem azt is, hogy mekkora valószínűséggel sorolja a tartalmat TV~Show-nak.

A vizsgált öt rekord esetében a modell mindegyiket helyesen osztályozta,
és a becsült valószínűségek:

\begin{itemize}
    \item sorozatoknál nagyon magasak (0.99 körül),
    \item filmeknél nagyon alacsonyak (0.00--0.03 között).
\end{itemize}

Ez azt mutatja, hogy a modell jól felismeri azokat a mintázatokat,
amelyek a két kategóriát megkülönböztetik: a sorozatoknál az évadok száma és bizonyos műfajok,
míg a filmeknél elsősorban a játékidő dominál.

\Section{Megjegyzések és továbbfejlesztés}

A modell teljesítménye kiváló, ugyanakkor további fejlesztési irányok is lehetségesek:

\begin{itemize}
    \item komplexebb modellek kipróbálása (Random Forest, XGBoost);
    \item hiperparaméter-optimalizálás;
    \item szöveges leírások (\code{description}) bevonása TF--IDF reprezentációval;
    \item a Netflix kategóriák inkonzisztenciáinak mélyebb vizsgálata.
\end{itemize}

A jelenlegi eredmények alapján a logisztikus regresszió megbízhatóan különbözteti meg a filmeket és a sorozatokat,
és jól értelmezhető alapot ad a további elemzésekhez.
