\chapter{Feltáró adatelemzés}

Az adattisztítási lépéseket követően a feldolgozott adathalmazon 
feltáró adatelemzést (\textit{Exploratory Data Analysis, EDA}) végeztem. 
A cél az volt, hogy megértsem az egyes jellemzők eloszlását, 
valamint az adatok közötti lehetséges összefüggéseket. 
Az elemzés során mind \textbf{egyváltozós}, mind \textbf{kétváltozós} vizsgálatokat készítettem, 
grafikus ábrázolások és statisztikai jellemzők segítségével.

\Section{Egyváltozós elemzés}

Az egyváltozós elemzés célja az volt, hogy feltárjam az egyes változók 
eloszlását, gyakoriságát és tipikus értékeit. 
Az elemzés során elsősorban a \texttt{type\_norm}, \texttt{release\_year}, 
\texttt{rating\_clean}, \texttt{country\_main}, valamint a 
\texttt{duration\_minutes} és \texttt{seasons} oszlopokat vizsgáltam.

\begin{itemize}
    \item \textbf{Típusok eloszlása:}  
    Az adatok alapján a Netflix kínálatában a filmek (\textit{Movie}) dominálnak, 
    a teljes katalógus mintegy kétharmadát teszik ki. 
    A fennmaradó rész sorozat (\textit{TV Show}). 
    Ez az arányos eloszlás fontos alapinformáció az osztályozási modellhez.
\end{itemize}

\begin{figure}[H]
    \centering
    \includegraphics[width=0.8\textwidth]{images/plot_type_counts.png}
    \caption{Típusok eloszlása a Netflix kínálatában (Movie vs TV Show)}
\end{figure}

\begin{itemize}
    \item \textbf{Megjelenési évek:}  
    A megjelenési év eloszlása azt mutatta, hogy a legtöbb tartalom 
    2010 után jelent meg, a csúcs 2018–2020 közé tehető. 
    A régebbi, 2000 előtti filmek száma viszonylag alacsony. 
    Ez jelzi, hogy a Netflix főként újabb produkciókat kínál.
\end{itemize}

\begin{figure}[H]
    \centering
    \includegraphics[width=0.8\textwidth]{images/plot_release_year_hist.png}
    \caption{Megjelenési év szerinti eloszlás}
\end{figure}

\begin{itemize}
    \item \textbf{Korhatár-besorolás (rating):}  
    A leggyakoribb minősítések a \textit{TV-MA} és a \textit{TV-14}, 
    amelyek felnőtt, illetve tinédzser korosztály számára ajánlott tartalmakat jelölnek. 
    Ez is alátámasztja, hogy a Netflix kínálata túlnyomórészt 
    érettebb közönségnek szóló produkciókból áll.
\end{itemize}

\begin{figure}[H]
    \centering
    \includegraphics[width=0.8\textwidth]{images/plot_rating_top10.png}
    \caption{Leggyakoribb korhatár-besorolások (Top 10)}
\end{figure}

\begin{itemize}
    \item \textbf{Országok eloszlása:}  
    A legtöbb cím az Egyesült Államokból származik, ezt követi India, 
    az Egyesült Királyság és Kanada. 
    A földrajzi megoszlás alapján jól látható, hogy a Netflix 
    globális tartalomszolgáltató, de az angol nyelvű produkciók túlsúlya egyértelmű.
\end{itemize}

\begin{figure}[H]
    \centering
    \includegraphics[width=0.8\textwidth]{images/plot_country_top15.png}
    \caption{A legtöbb tartalommal rendelkező országok (Top 15)}
\end{figure}

\Section{Kétváltozós elemzés}

A kétváltozós elemzés célja az volt, hogy megvizsgáljam, 
hogyan függnek össze egyes változók egymással. 
Két fő kapcsolatot elemeztem részletesen:

\begin{itemize}
    \item \textbf{Megjelenési év és tartalomtípus kapcsolata:} \\
    A filmek és sorozatok megjelenési éveit boxplot segítségével hasonlítottam össze. 
    Az eredmények alapján a sorozatok jellemzően újabbak: 
    medián megjelenési évük 2018, míg a filmeké 2016. 
    Ez arra utal, hogy a Netflix az utóbbi években inkább a sorozatok gyártására helyezte a hangsúlyt.
\end{itemize}

\begin{figure}[H]
    \centering
    \includegraphics[width=0.8\textwidth]{images/plot_year_box_by_type.png}
    \caption{A megjelenési év eloszlása típusonként (Movie vs TV Show)}
\end{figure}

Az elemzés eredményei megerősítették, hogy a filmek és sorozatok között nemcsak tartalmi, hanem időbeli és szerkezeti különbségek is megfigyelhetők.
Ezek a megállapítások szolgáltak alapul a következő fejezetben bemutatott \textbf{hipotézisvizsgálathoz} és \textbf{osztályozási modellhez}.
