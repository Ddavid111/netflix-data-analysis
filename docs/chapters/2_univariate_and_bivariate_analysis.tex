\chapter{Feltáró adatelemzés}

Az adattisztítási lépéseket követően a feldolgozott adathalmazon
feltáró adatelemzést (\textit{Exploratory Data Analysis, EDA}) végeztem.
A cél az volt, hogy megértsem az egyes jellemzők eloszlását,
valamint az adatok közötti lehetséges összefüggéseket.
Az elemzés során mind \textbf{egyváltozós}, mind \textbf{kétváltozós} vizsgálatokat készítettem,
grafikus ábrázolások és statisztikai jellemzők segítségével.

\Section{Egyváltozós elemzés}

Az egyváltozós elemzés célja az volt, hogy feltárjam az egyes változók
eloszlását, gyakoriságát és tipikus értékeit.
Az elemzés során elsősorban a \code{type\_norm}, \code{release\_year},
\code{rating\_clean}, \code{country\_main}, valamint a
\code{duration\_minutes} és \code{seasons} oszlopokat vizsgáltam.

\begin{itemize}
    \item \textbf{Típusok eloszlása:} \\
    Az adatok alapján a Netflix kínálatában a filmek (\textit{Movie}) dominálnak,
    a teljes katalógus mintegy kétharmadát teszik ki.
    A fennmaradó rész sorozat (\textit{TV Show}).
    Ez az arányos eloszlás fontos alapinformáció az osztályozási modellhez.
\end{itemize}

\begin{figure}[H]
    \centering
    \includegraphics[width=0.8\textwidth]{images/plot_type_counts.png}
    \caption{Típusok eloszlása a Netflix kínálatában (Movie vs.\ TV Show)}
\end{figure}

\begin{itemize}
    \item \textbf{Megjelenési évek:} \\
    A megjelenési év eloszlását két külön ábrán vizsgáltam:
    a 2000 előtti és a 2000 utáni tartalmakat szétválasztva.
    A 2000 előtti időszakban viszonylag kevés cím található, és ezek lassan növekvő tendenciát mutatnak.
    A 2000 utáni tartalmak száma viszont gyorsan emelkedik, különösen 2015 után.
    A legtöbb tartalom 2018–2020 között jelent meg, ami azt jelzi,
    hogy a Netflix kínálata erősen az újabb produkciókra épül.
\end{itemize}

\begin{figure}[H]
\centering
    \begin{minipage}{0.48\textwidth}
        \centering
        \includegraphics[width=\textwidth]{images/plot_release_year_before2000.png}
        \caption{\footnotesize Megjelenési év (2000 előtt)}
    \end{minipage}
    \hfill
    \begin{minipage}{0.48\textwidth}
        \centering
        \includegraphics[width=\textwidth]{images/plot_release_year_after2000.png}
        \caption{\footnotesize Megjelenési év (2000 után)}
    \end{minipage}
\end{figure}

\begin{itemize}
    \item \textbf{Korhatár-besorolás (rating):} \\
    A leggyakoribb minősítések a \textit{TV-MA} és a \textit{TV-14},
    amelyek felnőtt, illetve tinédzser korosztály számára ajánlott tartalmakat jelölnek.
    Ez arra utal, hogy a Netflix kínálatának jelentős része érettebb közönséget céloz.

    A korhatár-besorolásokat két módon is vizsgáltam:
    \begin{itemize}
        \item gyakoriság szerinti top--10 ábra,
        \item valamint egy logikai sorrendben (TV-Y $\rightarrow$ TV-MA) rendezett diagram,
              amely a besorolási skála szerkezetét is szemlélteti.
    \end{itemize}
\end{itemize}

\begin{figure}[H]
    \centering
    \includegraphics[width=0.8\textwidth]{images/plot_rating_top10.png}
    \caption{Leggyakoribb korhatár-besorolások (Top~10)}
\end{figure}

\begin{figure}[H]
    \centering
    \includegraphics[width=0.8\textwidth]{images/plot_rating_ordered.png}
    \caption{Korhatár-besorolások logikai sorrendben (TV-Y $\rightarrow$ TV-MA)}
\end{figure}

\begin{itemize}
    \item \textbf{Országok eloszlása:} \\
    A legtöbb cím az Egyesült Államokból származik, ezt követi India,
    az Egyesült Királyság és Kanada.
    A földrajzi megoszlás alapján jól látható, hogy a Netflix
    globális tartalomszolgáltató, de az angol nyelvű produkciók túlsúlya egyértelmű.
\end{itemize}

\begin{figure}[H]
    \centering
    \includegraphics[width=0.8\textwidth]{images/plot_country_top15.png}
    \caption{A legtöbb tartalommal rendelkező országok (Top~15)}
\end{figure}

\begin{figure}[H]
    \centering
    \includegraphics[width=0.95\textwidth]{images/plot_country_world_map.png}
    \caption{A Netflix tartalmak földrajzi eloszlása világtérképen}
\end{figure}

\Section{Kétváltozós elemzés}

A kétváltozós elemzés célja az volt, hogy megvizsgáljam,
hogyan függnek össze egyes változók egymással.
Két fő kapcsolatot elemeztem részletesen:

\begin{itemize}
    \item \textbf{Megjelenési év és tartalomtípus kapcsolata:} \\
    A filmek és sorozatok megjelenési évének eloszlását két külön hisztogramon ábrázoltam,
    ahol az átlagot és a mediánt is megjelenítettem.
    Az eredmények alapján mindkét kategóriában az utóbbi évek dominálnak,
    azonban a sorozatok megjelenési évei egyértelműen újabbak.
    A filmek medián megjelenési éve 2016, míg a sorozatoké 2018.
    Ez arra utal, hogy a Netflix az utóbbi időszakban egyre nagyobb hangsúlyt fektetett
    a sorozatok gyártására és fejlesztésére.
\end{itemize}

\begin{figure}[H]
    \centering
    \includegraphics[width=0.95\textwidth]{images/plot_year_hist_by_type.png}
    \caption{A megjelenési év eloszlása típusonként (Movie vs.\ TV Show)}
\end{figure}

Az elemzés eredményei megerősítették, hogy a filmek és sorozatok között nemcsak tartalmi,
hanem időbeli és szerkezeti különbségek is megfigyelhetők.
Ezek a megállapítások szolgáltak alapul a következő fejezetben bemutatott
\textbf{hipotézisvizsgálathoz} és \textbf{osztályozási modellhez}.
