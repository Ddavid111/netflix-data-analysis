\chapter{Adathalmaz gyűjtése és tisztítása}

A projekt során a \textbf{Netflix Movies and TV Shows Dataset} adathalmazt választottam,
amely a \textit{Kaggle} adatmegosztó platformon érhető el nyilvánosan:

\textbf{\href{https://www.kaggle.com/datasets/shivamb/netflix-shows}{https://www.kaggle.com/datasets/shivamb/netflix-shows}}.
\linebreak
Az adatbázis a Netflix streaming platformon elérhető filmeket és sorozatokat tartalmazza,
és több mint \textbf{8000 rekordot} foglal magában.
Minden sor egy-egy címet reprezentál, különböző jellemzőkkel, például:
\begin{itemize}
    \item a mű típusa (\textit{Movie} vagy \textit{TV Show}),
    \item a cím,
    \item a megjelenés éve,
    \item a hozzáadás dátuma a Netflix katalógusához,
    \item az ország, ahol készült,
    \item a rendező és a szereplők,
    \item a tartalom besorolása (pl. \textit{TV-MA}, \textit{PG-13}),
    \item a tartalom hossza (pl. ``90~min'' vagy ``2~Seasons''),
    \item valamint a műfajok a \code{listed\_in} mezőben (pl. ``Dramas, International Movies'').
\end{itemize}

\Section{Adatgyűjtés}
Az adathalmaz forrása egy nyilvános CSV-fájl (\code{netflix\_titles.csv}),
amelyet a fenti Kaggle hivatkozásról töltöttem le.
Az adatfájl jól strukturált, azonban több oszlopban szöveges típusú adatok szerepelnek,
ezért a statisztikai és gépi tanulási elemzésekhez előzetes előfeldolgozásra volt szükség.

\Section{Az adattisztítás célja}
Az adattisztítás célja az volt, hogy az adatok \textbf{konzisztens, hiánymentes és elemzésre alkalmas}
formában álljanak rendelkezésre,
és a későbbi elemzési, illetve modellezési lépések (pl. hipotézisvizsgálat, osztályozás)
ne ütközzenek formátumbeli vagy típushibákba.

A nyers adatok többféle problémát tartalmaztak, mint például hiányzó értékek (\code{NaN}),
különböző formátumok ugyanazon mezőn belül, illetve szöveges jellemzők,
amelyeket numerikus formára kellett alakítani.

\Section{Adattisztítási lépések}
Az adatok előkészítése több egymást követő lépésben történt:

\begin{enumerate}
    \item \textbf{Szöveges adatok egységesítése} \\
    Az összes szöveges mezőt megtisztítottam a felesleges szóközöktől és a kis-nagybetűs eltérésektől.
    A \code{NaN} vagy ``nan'' karakterláncokat valódi hiányzó értékként kezeltem.

    \item \textbf{Típusváltozó normalizálása} \\
    A \code{type} oszlop értékeit egységesítettem, így csak két kategória maradt:
    \code{Movie} és \code{TV Show}.
    Ez a későbbi osztályozási modell célváltozója lett.

    \item \textbf{Dátumok feldolgozása} \\
    A \code{date\_added} oszlopban szereplő szöveges dátumokat
    valós időbélyeggé alakítottam, ami lehetővé tette az időbeli trendek elemzését.

    \item \textbf{Tartalom hossza (duration)} \\
    A \code{duration} oszlop filmeknél perceket (pl. ``90~min''),
    sorozatoknál pedig évadszámot (pl. ``2~Seasons'') tartalmazott.
    Ezt a kétféle információt külön mezőkbe bontottam:
    \code{duration\_minutes} és \code{seasons}.
    A két mező diszjunkt módon jön létre, vagyis minden rekordnál csak az egyik tartalmaz értéket.

    \item \textbf{Szereplők száma} \\
    A \code{cast} mezőt feldolgozva létrehoztam a \code{cast\_count} változót,
    amely megszámolja, hány színész szerepel az adott tartalomnál.

    \item \textbf{Fő ország meghatározása} \\
    Azoknál a címeknél, ahol több ország szerepelt vesszővel elválasztva,
    csak az első országot tartottam meg (\code{country\_main}).

    \item \textbf{Korhatár-besorolás egységesítése} \\
    A \code{rating} oszlopban többféle formátumú besorolás volt.
    Ezeket egységesítettem a \code{rating\_clean} mezőbe.

    \item \textbf{Műfajok (genre) feldolgozása} \\
    A műfajok a \code{listed\_in} mezőben szerepeltek szövegként.
    Ezeket listává alakítottam, majd a 20 leggyakoribb műfajhoz bináris oszlopokat hoztam létre
    (pl. \code{genre\_Dramas}, \code{genre\_Comedies}),
    így a műfajok kvantitatív módon is bevonhatók lettek az elemzésbe.

    \item \textbf{Hiányzó értékek kezelése} \\
    A numerikus mezőkben mediánnal, a kategóriális mezőkben pedig a
    leggyakoribb értékkel pótoltam a hiányzó adatokat.

    \item \textbf{Tisztított adatok mentése} \\
    A tisztított adatokat \code{netflix\_cleaned.csv} néven mentettem el.
\end{enumerate}

\Section{Eredmény}
A tisztítás után az adathalmaz:
\begin{itemize}
    \item 8807 rekordot tartalmazott,
    \item 40 attribútummal rendelkezett,
    \item és mentes volt a típushibáktól, valamint a hiányzó kulcsváltozóktól.
\end{itemize}

