\chapter{Hipotézis megfogalmazása és tesztelése}

A feltáró adatelemzés során megfigyelhető volt, hogy a Netflix kínálatában 
a filmek és a sorozatok megjelenési évei eltérő eloszlást mutatnak. 
Úgy tűnt, hogy a sorozatok jellemzően újabbak, míg a filmek között 
gyakrabban találhatók korábbi produkciók. 
Ennek statisztikai vizsgálatához a következő hipotézist fogalmaztam meg.

\Section{Hipotézis megfogalmazása}

\begin{itemize}
    \item \textbf{Nullhipotézis ($H_0$):} A filmek és sorozatok megjelenési évének eloszlása között nincs szignifikáns különbség.
    \item \textbf{Alternatív hipotézis ($H_1$):} A filmek és sorozatok megjelenési évének eloszlása között szignifikáns különbség van.
\end{itemize}

Mivel a megjelenési év nem feltétlenül követi a normáleloszlást, 
nem-paraméteres tesztet alkalmaztam az összehasonlításhoz. 
A \textbf{Mann–Whitney U-próbát} választottam, amely két független mintát hasonlít össze 
anélkül, hogy eloszlásukra vonatkozó feltételezést tenne.

\Section{Tesztelés menete}

A próba során a két csoport a következő volt:
\begin{itemize}
    \item Filmek megjelenési évei
    \item Sorozatok megjelenési évei
\end{itemize}

A Pythonban futtatott Mann–Whitney U-teszt eredménye az alábbi volt:

\begin{verbatim}
Mann–Whitney U statisztika: 5820528.00 | p-érték: 2.051e-105
Filmek medián éve: 2016, Sorozatok medián éve: 2018
\end{verbatim}

\Section{Eredmények értelmezése}

A teszt p-értéke rendkívül kicsi ($p < 0.001$), 
ami azt jelenti, hogy elutasíthatjuk a nullhipotézist.  
Statisztikai szempontból tehát szignifikáns különbség van a filmek és sorozatok megjelenési évei között.  
A mediánértékek alapján a sorozatok jellemzően \textbf{újabb megjelenésűek}, 
ami összhangban van a Netflix stratégiájával, miszerint az utóbbi években 
főként saját gyártású sorozatokkal bővítette kínálatát.

\begin{figure}[H]
    \centering
    \includegraphics[width=0.8\textwidth]{images/plot_year_box_by_type.png}
    \caption{A megjelenési év eloszlása filmek és sorozatok között}
\end{figure}

\Section{Következtetés}

A statisztikai vizsgálat megerősítette, hogy a filmek és sorozatok megjelenési éve 
szignifikánsan eltér egymástól. 
Ez azt sugallja, hogy a Netflix az elmúlt években a sorozatokra helyezte a hangsúlyt, 
míg a filmkínálat inkább korábbi időszakok produkcióiból áll. 
Ez az eredmény fontos alapot biztosít a következő fejezetben ismertetett 
\textbf{osztályozási modell} kialakításához.
