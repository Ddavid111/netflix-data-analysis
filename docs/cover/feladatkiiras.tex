\chapter*{Feladatleírás}
\addcontentsline{toc}{chapter}{Feladatleírás}

\section*{Célkitűzés}
A féléves feladat célja egy nyilvános adathalmazból történő adatgyűjtés, tisztítás, feltáró adatelemzés, hipotézisvizsgálat, valamint klaszterezés illetve osztályozás elvégzése.

\section*{Feladat részletezése}

\subsection*{1. Adathalmaz gyűjtése és tisztítása}
\begin{itemize}
    \item Keressen az interneten egy tetszőleges adathalmazt, majd töltse le.
    \item Végezze el az elemzés előkészítéséhez szükséges adattisztítási lépéseket.
\end{itemize}

\subsection*{2. Feltáró adatelemzés}
\begin{itemize}
    \item Készítsen \textbf{egyváltozós elemzést} az adathalmazról az egyes változók eloszlásának megértéséhez.
    \item Végezzen \textbf{kétváltozós elemzést} két kiválasztott változó kapcsolatának vizsgálatára.
\end{itemize}

\subsection*{3. Hipotézis megfogalmazása és tesztelése}
\begin{itemize}
    \item Fogalmazzon meg egy, az adathalmazhoz kapcsolódó releváns hipotézist.
    \item Tesztelje a hipotézist megfelelő statisztikai módszerekkel, majd értelmezze az eredményeket.
\end{itemize}

\subsection*{4. Klaszterezés vagy osztályozás}
\begin{itemize}
    \item Készítsen klaszterező vagy osztályozó modellt a tisztított adathalmazon a Python gépi tanulási algoritmusainak felhasználásával. Számítsa ki a teljesítménymutatókat.
    \item Alkalmazza a modellt új adatokon.
\end{itemize}

\subsection*{5. Leadási és bemutatási követelmények}
A Teams feladathoz csatolva az alábbi fájlokat kell benyújtani:
\begin{itemize}
    \item Python forráskód.
    \item Az elemzéshez felhasznált adathalmaz.
    \item Egy PDF jelentés, amely tartalmazza:
    \begin{itemize}
        \item Az adathalmaz forrását (link) és leírását.
        \item Az adattisztítás lépéseinek részletezését.
        \item A feltáró elemzések összefoglalását.
        \item A hipotézis megfogalmazását és a tesztelés eredményeit.
        \item A klaszterezéshez vagy osztályozáshoz használt gépi tanulási algoritmus leírását.
    \end{itemize}
    \item A klaszterező / osztályozó modell teljesítményét és az új adatokra való alkalmazásának eredményeit.
\end{itemize}

\section*{Prezentáció}
Nappali tagozaton a kész feladatot személyesen is be kell mutatni, összefoglalva a főbb eredményeket,
a módszertant és a következtetéseket.

\section*{Leadási határidő}
A feladat leadási határideje MS Teams-ben: \textbf{2025. november 20.} \\
